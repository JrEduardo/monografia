\documentclass[10pt,ignorenonframetext,]{beamer}
\setbeamertemplate{caption}[numbered]
\setbeamertemplate{caption label separator}{:}
\setbeamercolor{caption name}{fg=normal text.fg}
\usepackage{amssymb,amsmath}
\usepackage{ifxetex,ifluatex}
\usepackage{fixltx2e} % provides \textsubscript
\usepackage{lmodern}
\ifxetex
  \usepackage{fontspec,xltxtra,xunicode}
  \defaultfontfeatures{Mapping=tex-text,Scale=MatchLowercase}
  \newcommand{\euro}{€}
\else
  \ifluatex
    \usepackage{fontspec}
    \defaultfontfeatures{Mapping=tex-text,Scale=MatchLowercase}
    \newcommand{\euro}{€}
  \else
    \usepackage[T1]{fontenc}
    \usepackage[utf8]{inputenc}
      \fi
\fi
% use upquote if available, for straight quotes in verbatim environments
\IfFileExists{upquote.sty}{\usepackage{upquote}}{}
% use microtype if available
\IfFileExists{microtype.sty}{\usepackage{microtype}}{}

% Comment these out if you don't want a slide with just the
% part/section/subsection/subsubsection title:
\AtBeginPart{
  \let\insertpartnumber\relax
  \let\partname\relax
  \frame{\partpage}
}
\AtBeginSection{
  \let\insertsectionnumber\relax
  \let\sectionname\relax
  \frame{\sectionpage}
}
\AtBeginSubsection{
  \let\insertsubsectionnumber\relax
  \let\subsectionname\relax
  \frame{\subsectionpage}
}

\setlength{\parindent}{0pt}
\setlength{\parskip}{6pt plus 2pt minus 1pt}
\setlength{\emergencystretch}{3em}  % prevent overfull lines
\setcounter{secnumdepth}{0}
%% ======================================================================
%% Eduardo Junior
%% eduardo.jr@ufpr.br
%% 22-11-2015
%% ======================================================================
%% Template latex beamer para ser usado na opção `include: in_header:`
%% de arquivos Rmd com ouput beamer_template

%% ======================================================================
%% README

% O YAML do documento Rmd usado é, modificações podem alterar os
% resutados obtidos.
% ---
% date: "`r format(Sys.time(), '%d de %B de %Y')`"
% fontsize: "10pt"
% output: 
%   beamer_presentation: 
%     fig_caption: yes
%     highlight: zenburn
%     includes:
%       in_header: preambulo-beamer.tex
%     keep_tex: yes
%     slide_level: 3
% ---

%% ======================================================================
%% Lista de pacotes para tarefas gerais no documento

\usepackage{xcolor}

\usepackage{setspace}
\usepackage{tikz}
\usepackage{pgfgantt}
\usetikzlibrary{backgrounds}
\definecolor{done}{RGB}{120, 180, 120}
\definecolor{do}{RGB}{180, 120, 120}

%% ======================================================================
%% Tema e cores do documento

\usetheme{CambridgeUS}
\setbeamertemplate{itemize items}[triangle]
\setbeamertemplate{navigation symbols}{}

\setbeamertemplate{frametitle}{
  \nointerlineskip
  \begin{beamercolorbox}[sep=0.3cm, ht=1.8em, 
    wd=\paperwidth]{frametitle}
    \vbox{}\vskip-2ex%
    \strut\hspace*{3ex}\large\bfseries\insertframetitle\strut
    \vskip-0.8ex%
  \end{beamercolorbox}
}

\setbeamercolor{frametitle}{fg=teal}
\setbeamercolor{structure}{fg=teal}
\setbeamercolor{palette primary}{bg=gray!30, fg=teal}
\setbeamercolor{palette tertiary}{bg=teal, fg=white}
\setbeamercolor{footlinecolor}{fg=white,bg=teal}
\setbeamercolor{caption name}{fg=teal}

%% ======================================================================
%% Página Inicial

\setbeamertemplate{title page}[default]
\setbeamercolor{title}{fg=teal}
\setbeamercolor{author}{fg=black!70}
\setbeamercolor{institute}{fg=black!70}
\setbeamercolor{date}{fg=black!70}
\setbeamerfont{title}{series=\bfseries}

%% ======================================================================
%% Definição do cabeçalho e rodapé

\setbeamertemplate{headline}{\bfseries
  \leavevmode%
  \hbox{%
    \begin{beamercolorbox}[wd=.5\paperwidth, ht=2.2ex, dp=1ex, right, 
      rightskip=1em]{section in head/foot}
      \hspace*{2ex}\insertsectionhead
    \end{beamercolorbox}%
    \begin{beamercolorbox}[wd=.5\paperwidth, ht=2.2ex, dp=1ex, left,
      leftskip=1em]{subsection in head/foot}
      \insertsubsectionhead\hspace*{2ex}
    \end{beamercolorbox}}
  \vskip0pt
}

\makeatletter
\setbeamertemplate{footline}{\ttfamily\bfseries
  \leavevmode%
  \hbox{%
    \begin{beamercolorbox}[wd=.3\paperwidth, ht=2.2ex, dp=1ex, right, 
      rightskip=1em]{footlinecolor}
      \insertshortauthor%
    \end{beamercolorbox}%
    \begin{beamercolorbox}[wd=.6\paperwidth, ht=2.2ex, dp=1ex, left,
      leftskip=1em]{footlinecolor}
      \hfill\insertshorttitle%
    \end{beamercolorbox}%
    \begin{beamercolorbox}[wd=.1\paperwidth, ht=2.2ex, dp=1ex, left,
      leftskip=1em]{footlinecolor}
      Slide \insertframenumber
    \end{beamercolorbox}}
  \vskip0pt
}
\makeatother

%% ======================================================================
%% Fontes

\usepackage{palatino}
\usepackage{eulervm}
\usepackage[none]{ubuntu}
\usepackage{verbatim}

\usefonttheme{professionalfonts}
\usefonttheme{serif}
\renewcommand{\ttdefault}{ubuntumono}
\renewcommand{\ttfamily}{\fontUbuntuMono}
\makeatletter
\def\verbatim@font{\small\fontUbuntuMono}
\makeatother

%% ======================================================================
%% Colorindo ambiente verbatim

\usepackage{etoolbox}
\makeatletter
\preto{\@verbatim}{\topsep=0pt \partopsep=0pt }
\makeatother

\let\oldv\verbatim
\let\oldendv\endverbatim

\def\verbatim{\setbox0\vbox\bgroup\oldv}
\def\endverbatim{\oldendv\egroup \hspace*{-0.6ex}\colorbox[gray]{0.93}{\usebox0}}
% \def\endverbatim{\oldendv\egroup \usebox0}


%% ======================================================================
%% Layout do tableofcontents

\setbeamertemplate{section in toc}{
  {\color{teal} \bfseries\inserttocsectionnumber.}~
  {\color{black}\inserttocsection}
}

%% ======================================================================
%% Formatando slides para seções e subseções

\AtBeginSection{
  \let\insertsectionnumber\relax
  \let\sectionname\relax
  \frame{\frametitle{Sumário}
    \tableofcontents[currentsection, subsectionstyle=hide]
  }
}

\AtBeginSubsection{
  \let\insertsubsectionnumber\relax
  \let\subsectionname\relax
}

%% ======================================================================
%% Metadados do documento

\title[Extensões e Aplicações do Modelo COM-Poisson]{Extensões e
  Aplicações do Modelo de Regressão Conway-Maxwell-Poisson para
  Modelagem de Dados de Contagem} 
\author[Eduardo E. R. Junior \& Walmes M. Zeviani]{
  Eduardo Elias Ribeiro Junior \\
  Orientação: Prof. Dr. Walmes Marques Zeviani}
\institute[]{
  Projeto de Pesquisa - Laboratório A\\
  Departamento de Estatística (DEST)\\
  Universidade Federal do Paraná (UFPR)
}

%% ======================================================================
%% TODO

% * Realizar as definidos em um documento template, para ser usado no YAML
% como `template: template.tex`. Assim não são sobrepostas as definição
% padrão do knitr com as modificações feitas. Utilizar como base o arquivo
% em ~/.../rmarkdown/rmd/latex/default.tex

\date{23 de novembro de 2015}

\begin{document}

\begin{frame}{}

\titlepage

\end{frame}

\begin{frame}{Sumário}

\tableofcontents[hideallsubsections]

\end{frame}

\section{Introdução}\label{introducao}

\subsection{Modelos de regressão}\label{modelos-de-regressao}

\begin{frame}{}

\end{frame}

\subsection{Modelos para dados de
contagem}\label{modelos-para-dados-de-contagem}

\begin{frame}{}

\end{frame}

\subsection{Distribuição COM-Poisson}\label{distribuicao-com-poisson}

\begin{frame}{}

\end{frame}

\subsection{Modelo de regressão
COM-Poisson}\label{modelo-de-regressao-com-poisson}

\begin{frame}{}

\end{frame}

\subsection{Extensões do modelo de regressão
COM-Poisson}\label{extensoes-do-modelo-de-regressao-com-poisson}

\begin{frame}{}

\end{frame}

\section{Objetivos}\label{objetivos}

\begin{frame}{}

\end{frame}

\section{Materiais e Métodos}\label{materiais-e-metodos}

\subsection{Resursos Computacionais}\label{resursos-computacionais}

\begin{frame}{}

\end{frame}

\subsection{Conjunto de dados}\label{conjunto-de-dados}

\begin{frame}{}

\end{frame}

\subsection{Estimação por máxima
verossimilhança}\label{estimacao-por-maxima-verossimilhanca}

\begin{frame}{}

\end{frame}

\subsection{Critérios para Comparação}\label{criterios-para-comparacao}

\begin{frame}{}

\end{frame}

\section{Cronograma}\label{cronograma}

\begin{frame}{}

\begin{figure}
\centering
\hspace{-0.5cm}
\begin{tikzpicture}[thick, scale=0.75, every node/.style={scale=0.75}]
\begin{ganttchart}[
canvas/.append style={fill=none},
y unit title=0.5cm, % Size da indicação do tempo
y unit chart=0.7cm, % Size do altura das colunas
x unit= 5.8mm, % largura das celulas
vgrid={*1{black!50, dotted}}, % grid cinza vertical
hgrid={*1{black!50, dotted}}, % grid cinza horizontal
title height=1, % Size dos dias
bar/.style={fill=done, draw=teal},
bar incomplete/.append style={fill=gray!10, draw=teal}, 
bar label node/.append style={align=right},
bar label font=\scriptsize\color{black!65},
group label font=\bfseries\scriptsize\color{black},
group left peak width=0.2,
group right peak width=0.2,
group left peak height=0.15,
group right peak height=0.15,
group left shift=.1, group right shift=-.1,
bar height=0.6, % size das barras de tarefas
bar left shift=.2, bar right shift=-.2,
bar top shift=.1, bar height=.8,
link/.style={-to, line width=0.7pt, black!50},
link type=dr
]{1}{24} % 
%\gantttitle{\color{teal}\bf Cronograma de atividades para 2016}{24}\\
\gantttitle{Fevereiro}{4}
\gantttitle{Março}{4} 
\gantttitle{Abril}{4} 
\gantttitle{Maio}{4} 
\gantttitle{Junho}{4} 
\gantttitle{Julho}{4} \\
%\gantttitlelist{1,...,24}{1} \\
\ganttgroup[group label node/.append style={align=right}, 
progress=75]{Projeto \ganttalignnewline de Pesquisa}{1}{3} \\
\ganttbar[progress=80]{Redação da \ganttalignnewline versão 
final}{1}{2} \\
\ganttbar[progress=0, progress label text=]{Entrega à
\ganttalignnewline banca}{3}{3} \\
\ganttgroup[group label node/.append style={align=right},
progress=15]{Elaboração \ganttalignnewline da Pesquisa}{2}{20} \\
\ganttbar[progress=40]{Revisão de \ganttalignnewline
literatura}{2}{8} \\
\ganttbar[progress=10]{Implementação
\ganttalignnewline Computacional}{7}{14} \\
\ganttbar[progress=0, progress label text= ]{Análise
\ganttalignnewline dos dados}{11}{14} \\
\ganttbar[progress=0, progress label text= ]{Discussão
\ganttalignnewline dos resultados}{13}{16} \\
\ganttbar[progress=0, progress label text= ]{Redação
\ganttalignnewline da pesquisa}{15}{20} \\
\ganttgroup[group label node/.append style={align=right},
progress=0, progress label text=]{Defesa \ganttalignnewline do
Trabalho}{20}{21} \\  
\ganttbar[progress=0, progress label text= ]{Elaboração da
\ganttalignnewline apresentação}{20}{20} \\
\ganttgroup[group label node/.append style={align=right},
progress=0, progress label text=]{Revisão Final}{21}{24} \\ 
\ganttbar[progress=0, progress label text= ]{Incorporação
\ganttalignnewline das sugestões}{21}{22} \\
\ganttbar[progress=0, progress label text= ]{Redação da 
\ganttalignnewline versão final}{22}{23}
 \ganttlink{elem1}{elem2}
 \ganttlink{elem1}{elem3}
 \ganttlink{elem4}{elem5}
 \ganttlink{elem4}{elem7}
 \ganttlink{elem4}{elem8}
 \ganttlink{elem5}{elem6}
 \ganttlink{elem5}{elem7}
 \ganttlink{elem8}{elem10}
 \ganttlink{elem10}{elem12}
 \ganttlink{elem10}{elem13}
 \ganttlink{elem12}{elem13}
\end{ganttchart}
\end{tikzpicture}
\caption{Cronograma de atividades para 2016}
\end{figure}

\end{frame}

\end{document}
